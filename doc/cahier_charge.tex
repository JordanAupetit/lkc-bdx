\documentclass[16pts]{report}
\usepackage[utf8]{inputenc}
\usepackage[T1]{fontenc}
\usepackage[francais]{babel}
\usepackage{xcolor}
\usepackage[hyphens]{url}
\usepackage[hidelinks]{hyperref}
\usepackage{amsmath}
\usepackage{graphicx}
\usepackage{geometry}
\usepackage{textcomp}
\hypersetup{hypertexnames=true}
\geometry{hmargin=2.5cm,vmargin=1.5cm}

%\maketitle
%\clearpage

\begin{document}
\bibliographystyle{unsrt}
\nocite{*}

\chapter{Besoins fonctionnels}
\label{cha:Besoins fonctionnels}


\section{Mots-clés et description d'une option}
\label{sec:Mots-clés et description d'une option}


Un utilisateur doit pouvoir sélectionner une option et y ajouter un mot-clé.
Par exemple, il peut spécifier qu’une option possède le mot-clé “Network”. Il
pourra également supprimer et modifier les mots-clés existants. De la même
façon, l’utilisateur pourra éditer la description d’une option.


\section{Recherche}
\label{sec:Recherche}

L’utilisateur doit pouvoir effectuer une recherche sur le nom, la description
et les mots-clés d'une option lors de la configuration du noyau. La fonction de
recherche ne posera pas de véritables problèmes techniques. L’utilisateur
pourra faire défiler les lignes contenant les occurrences du mot recherché.

\section{Détection du matériel}
\label{sec:Détection du matériel}

La génération d’une configuration prédéfinie par détection du matériel hôte est
un besoin fonctionnel. En effet, un utilisateur ne voulant que le minimum
nécessaire dans son noyau est confronté à ce soucis de recherche et de
correspondance de module au matériel.  Il existe des outils Linux permettant de
récupérer la liste du matériel associé aux bus PCI et USB de la machine hôte.
L’étude des fichiers des répertoires “/dev /sys” et “/proc” est donc nécessaire
afin d’acquérir de plus amples informations, car nous n’avons pas trouvé
d’outils Linux permettant de récupérer directement le ou les modules noyau
utilisés sur une entrée.

Afin de rendre la correspondance des modules du noyau Linux au matériel plus
pertinente et fiable, l’idée serait de récupérer des logs d’exécution du scan
du matériel afin de peupler de manière volontaire une base de donnée
communautaire.

\section{Génération du fichier .config}
\label{sec:Génération du fichier .config}

L’utilisateur doit pouvoir “générer” où il le souhaite le fichier de
configuration généré par l’application. Un message (sous la forme d’une pop-up)
avertira l’utilisateur de la présence de conflits ou d’erreur de dépendances
avant que la génération soit faite.

\section{Gestion des conflits}
\label{sec:Gestion des conflits}

L’application doit pouvoir “gérer les conflits et les dépendances” entre les
options de configuration. Grâce au fichier kconfig, l’application pourra, pour
chaque option de configuration, connaître ses dépendances et les options avec
lesquelles elle rentre en conflit. De ce fait, chaque fois que l’utilisateur
sélectionne ou déselectionne une option de configuration, des tests sont
effectués par l'application pour savoir si cette option entre en conflit avec
une autre (si elle est cochée) ou génère une erreur de dépendance (si elle est
décochée).

\section{Chargement d'un fichier .config}
\label{sec:Chargement d'un fichier .config}

La reprise d’un fichier .config doit permettre à un utilisateur de charger une
configuration existante. Le chargement d’un fichier .config corrompu provoquera
une erreur, et s’il y a des options inconnues, elles seront simplement
ignorées. Cela impliquera le déclenchement du mécanisme de résolution de
conflits.

\section{Plateforme de partage communautaire}
\label{sec:Plateforme de partage communautaire}

L’utilisateur peut (si il le souhaite) envoyer sur la plateforme de partage, la
correspondance entre le materiel et le module capable de le faire fonctionner,
enrichissant ainsi la base de donnée de la plateforme.  De la même façon, lors
de la détection du matériel, l’application pourra trouver les modules à activer
(dans le fichier .config) grâce à cette plateforme. Cette plateforme servira
également à stocker les “descriptions” et les “mots-clés” des options.

\chapter{Besoins non fonctionnnels}
\label{cha:Besoins non fonctionnnels}

\section{Facilité d'utilisation}
\label{sec:Facilité d'utilisation}

Notre projet consiste à améliorer l’utilisation des fonctionnalités de base des
outils existants, en les rendant plus accessibles. On a pu constater que le
système de recherche des options n’est pas simple à prendre en main et que la
gestion des conflits pouvait être améliorée.

Nous avons donc décidé d’améliorer la recherche en cherchant au sein des
descriptions des options et pas seulement dans leur nom. Et au lieu de cacher
les options pouvant créer un conflit, nous les affichons mais avec des
indications sur la provenance du conflit, ce qui permet à l’utilisateur de
pouvoir trouver cette option et de résoudre ce conflit s’il le souhaite.

\section{Profil utilisateur}
\label{sec:Profil utilisateur}

Actuellement, les outils de configuration d’un noyau Linux sont “réservés” aux
personnes averties. Il faut donc que les fonctionnalités recherchées soient
présentes. Ces améliorations permettront de toucher un plus large public, de
débutant à expert.

\section{Portabilité}
\label{sec:Portabilité}

Il y a deux aspects lié à la portabilité. Dans un premier temps, le fait qu’il
est possible que l’environnement dans lequel l’application sera exécutée ne
possèdera pas de serveur X (interface graphique). Par conséquent, une solution
privilégiant une interface console (comme Ncurse) serait donc intéressante
puisqu’elle conviendrait à la fois à l’utilisation de l’application sur un
serveur et sur un ordinateur personnel (ou même sur tout type de support
supportant un noyau Linux). Cependant, l’application se voulant être tout
public, une interface (lancée depuis la console) pourrait se montrer un peu
austère pour un utilisateur habitué à une interface graphique plus
“traditionnelle”.  Dans un second temps, l’application pourra être
fonctionnelle sur un système d’exploitation Windows.

\section{Contrainte légales}
\label{sec:Contrainte légales}

Cet outil sera open source. De ce fait, il va utiliser la licence GPLv3. Cela
permettra la reprise éventuelle de ce projet dans l’avenir.

\section{Interface web}
\label{sec:Interface web}

L’utilisateur devrait pouvoir réaliser la détection de son matériel à partir
d’une interface web (comme le site : ma-config.com).

\end{document}
