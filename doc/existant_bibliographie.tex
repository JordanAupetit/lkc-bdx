\documentclass[16pts]{report}
\usepackage[utf8]{inputenc}
\usepackage[T1]{fontenc}
\usepackage[francais]{babel}
\usepackage{xcolor}
\usepackage[hyphens]{url}
\usepackage[hidelinks]{hyperref}
\usepackage{amsmath}
\usepackage{graphicx}
\usepackage{geometry}
\usepackage{textcomp}
\hypersetup{hypertexnames=true}
\geometry{hmargin=2.5cm,vmargin=1.5cm}

%\maketitle
%\clearpage

\begin{document}
\bibliographystyle{unsrt}
\nocite{*}

\chapter{Contexte}
\label{cha:Contexte}

Il est possible de compiler soi-même un noyau Linux. Cela permet de créer un
noyau répondant uniquement aux besoins de l’utilisateur. La tâche peut être
très fastidieuse, car les outils permettant de modifier le fichier de
configuration ne sont pas simples d’utilisation et il est difficile de trouver
les options que l’on aimerait avoir.  Ce projet a pour but de proposer un outil
permettant à un utilisateur non expert de configurer simplement son noyau en
l'allégeant à sa guise.

\chapter{Configuration des options du noyau}
\label{cha:Configuration des options du noyau}

La principale tâche à effectuer avant de lancer la compilation du noyau est de
créer un fichier “.config” comportant toutes les options disponibles pour le
noyau.  Après avoir récupéré le noyau (sur Kernel.org) on constate qu’il n’y a
pas de fichier “.config” par défaut. Il faut donc le générer et plusieurs
options s’offrent à nous :

Récupérer le fichier “.config” d’un noyau sur l’ordinateur que l’on souhaite,
mais il risque d’y avoir des incompatibilités à cause de la version des noyaux.

\begin{description}
    \item[make oldconfig :] Qui permet de récupérer la configuration du noyau
        courant et corrige le problème précédent, car en cas d’options
        différentes, le système demandera à l’utilisateur de choisir. Il sera
        difficile d’optimiser ce fichier dans le cas où le noyau courant
        contient des options inutiles.

    \item[make defconfig :] Qui permet de générer un fichier de configuration
        minimale.  C’est la meilleure option permettant d’optimiser son noyau
        sans repartir de zéro.  Malgré tout, il crée une configuration
        "minimale" générique et non adaptée à la machine de l’utilisateur.
        Le fichier peut donc être optimisé.
\end{description}

Après avoir généré une configuration initiale, il est possible d’utiliser
différents outils pour la modifier tels que :

\begin{description}
    \item[make config]      	Un programme en ligne de commande \\
        \includegraphics[scale=0.7]{illustrations/configLine.png} \pagebreak
    \item[make menuconfig]      Un programme utilisant ncurse \\
        \includegraphics[scale=0.7]{illustrations/menuconfig.png} \\
    \item[make xconfig]     	Un programme utilisant QT \\
        \includegraphics[scale=0.4]{illustrations/xconfig.png} \pagebreak
    \item[make gconfig] 	    Un programme utilisant GTK \\
        \includegraphics[scale=1]{illustrations/gconfig.jpg} \\
    \item[eCos], qui permet de configurer les noyaux pour le système
        d’exploitation eCos. C’est une autre source d’information
        sur laquelle se baser. \\
        \includegraphics[scale=1.3]{illustrations/eCos_config.png} \pagebreak
    \item[kcheck (kernel check)] permet également de configurer les options d’un noyau
        à compiler.
        Il propose deux modes :

    \begin{description}
        \item[Automatique :] kcheck va tenter de déterminer les options du kernel
            en fonction de la machine sur laquelle il est lancé
        \item[Manuel :] kcheck permet à l’utilisateur de modifier comme bon lui
            semble les différentes options du fichier de configuration du kernel.
    \end{description}
        \includegraphics[scale=0.8]{illustrations/kernel_check.png}\\
\end{description}

Une étude a été réalisée par Kacper Bak et Karim Ali de l’université Waterloo
afin d’améliorer la convivialité de la configuration d’un noyau Linux
\cite{Waterloo:Etude}.

Celle-ci a permis de mieux cerner les besoins des utilisateurs en observant
leurs difficultés à utiliser les outils existants. Ils ont réalisé un prototype
qu’ils ont fait tester et qu’ils ont amélioré afin d’obtenir le squelette d’un
outil facilitant la configuration d’un noyau Linux. \\

\includegraphics[scale=0.6]{illustrations/lkc_config.png} \\

Leur projet a donc abouti sur un prototype non fonctionnel.  L’ergonomie de
leur application sera une grande source d’inspiration pour nous de par les avis
qu’ils ont récoltés auprès d’utilisateurs.

Les quatre premiers outils présentés ("config", "menuconfig", "xconfig" et
"gconfig") sont les plus couramment utilisés pour configurer un noyau Linux
avant sa compilation. Nous sommes arrivés à cette conclusion, car dans un
premier temps, ces outils sont présents au sein même des sources du noyau
Linux.  Et dans un second temps, en recherchant comment d'autres personnes
réalisaient cette configuration, nous avons constaté qu'ils utilisaient à
chaque fois l'un de ces outils et principalement ceux avec une interface
graphique.

Pour ce qui est des deux derniers outils ("eCos", et "kCheck"), ils sont
beaucoup moins répandus, mais ils répondent à différents besoins de notre projet
dont nous allons nous inspirer :
\begin{itemize}
    \item La gestion des conflits entre les options
    \item La détection du matériel pour générer le fichier de configuration
\end{itemize}

Enfin, le dernier outil est seulement un prototype réalisé par l'université de
Waterloo, mais qui constitue notre plus importante base pour ce projet. En
effet, celui-ci ayant été réalisé après une étude des besoins des utilisateurs,
il comporte des informations précieuses pour réaliser notre PdP.

\chapter{Difficultés de la configuration}
\label{cha:Difficultés de la configuration}


Il existe plusieurs raisons qui font que la configuration des options d’un
noyau est une tâche fastidieuse et difficile :

Tout d’abord, le principal problème se situe au niveau des conflits et des
dépendances entre les options. En effet, celles-ci peuvent être liées à
d’autres options ou même exclusives, c’est-à-dire que si une option est
sélectionnée, une autre peut ne plus l’être.  Le problème de certains des
outils actuels est que les options en conflit avec les options actuelles ne
sont plus visibles. Donc lorsque l’on cherche une option précise qui n’est plus
affichée et que l’on ne sait pas quelle précédente option est en conflit avec
elle, il est très difficile de corriger cette erreur.

En outre, lorsqu'une option est sélectionnée, les outils actuels désactivent
automatiquement et sans prévenir l'utilisateur celles qui sont en conflit ou
celles qui en dépendent.

Enfin, il est compliqué pour un utilisateur non expert de trouver une option
précise sans la connaître parfaitement. Le nom de cette option ne représente
pas toujours très bien sa fonction, ce qui fait que la recherche des options
dans l’outil n’est pas aisée.  On constate qu’il y a une aide pour chacune des
options et il est dommage que la recherche par mots-clés ne s’effectue pas
également sur l’aide des fonctions.

\chapter{Recherches bibliographiques}

Notre sujet se repose sur un projet existant. En effet, l'université de
Waterloo a réalisé une étude approfondit \cite{Waterloo:Etude} sur la facilité
d'utilisation des outils de configuration du noyau Linux. On y trouve les
résultats de leurs tests auprès de différents utilisateurs, ce qui nous permet
d'avoir des informations sur les besoins réels vis-à-vis de cet outil. L'équipe
a réalisé un prototype qui prend en compte ces modifications et celui-ci peut
être trouvé dans leur dépôt Github \cite{Waterloo:Github}.

Nous avons trouvé d'autres avis d'utilisateurs au sein d'une étude
\cite{Hubaux:2012:USC:2110147.2110164}. On peut y observer les principaux
problèmes qu'ils ont rencontrés. Par exemple, on peut constater que la gestion
des conflits est une fonctionnalité qui pose généralement des difficultés.

Pour mieux comprendre notre sujet, nous avons configuré et compilé un noyau
Linux que nous avons trouvé sur le site officiel \cite{Kernel}. On peut trouver
au sein du dossier téléchargé, différents outils permettant de réaliser la
configuration du noyau avant sa compilation. On y trouve "menuconfig",
"xconfig" et "gconfig" qui sont des outils graphiques plus simple d'utilisation
que l'outil en ligne de commande "config".

La configuration étant difficile, nous avons trouvé sur le forum de Linux
\cite{Existant:Kernel:ForumTutoConfig} des explications très détaillées. Ce
guide reprend "pas à pas" chacune des options et les détaille une à une afin
de mieux comprendre ce qui peut être activé ou non.

Une fonctionnalité qui pourrait être utile pour un utilisateur serait de
pouvoir détecter le matériel de son ordinateur (ou une partie) afin de pouvoir
générer le fichier de configuration correspondant à sa machine. Nous avons
trouvé un tutoriel \cite{Existant:Kernel:outils} traitant de la compilation du
noyau Linux et qui évoque ce point particulier.

Un des problèmes évoqués par les testeurs de l'étude \cite{Waterloo:Etude} de
l'université de Waterloo, est que le système de gestion des conflits des outils
actuels n'est pas pratique, car aucune indication n'est donnée lorsqu'une
option est sélectionnée. Nous avons trouvé un outil qui effectue ce traitement
des conflits, mais pour un système d'exploitation différent : eCos
\cite{Existant:EcosConfig}. En nous en inspirant, nous pourrons éventuellement
proposer un mécanisme similaire dans notre outil de configuration.


L'étude \cite{Waterloo:Etude} explique brièvement comment fonctionne
l'arborescence des modules au sein du noyau Linux. Nous avons donc décidé
d'approfondir ce point et nous avons trouvé un fichier
\cite{Existant:Kconfig:frontends} expliquant plus en détail le fonctionnement
pour l'outil "menuconfig". En reprenant le même mécanisme utilisé par les
outils officiels, nous pourrons éventuellement nous abstenir de refaire le
parsage des nombreux dossiers des sources. Nous avons également récupéré de la
documentation \cite{Existant:Kconfig:vueDensemble}
\cite{Existant:Kconfig:langage} qui contient les éléments de syntaxe qui nous
ont aidés à comprendre comment s'effectue la génération du fichier ".config".
De plus, on trouve sur le site officiel du noyau Linux des indications
\cite{Existant:Kconfig:modules} permettant de compiler un module externe au
sein du noyau. Celui-ci nous permet donc de mieux comprendre le fonctionnement
des fichiers "Kbuild".

\bibliography{./bibliographie}
\end{document}
