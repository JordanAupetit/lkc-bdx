\documentclass[16pts]{report}
\usepackage[utf8]{inputenc}
\usepackage[T1]{fontenc}
\usepackage[francais]{babel}
\usepackage{xcolor}
\usepackage[hyphens]{url}
\usepackage[hidelinks]{hyperref}
\usepackage{amsmath}
\usepackage{graphicx}
\usepackage{geometry}
\usepackage{textcomp}
\hypersetup{hypertexnames=true}
\geometry{hmargin=2.5cm,vmargin=1.5cm}

\usepackage{float} %Option H pour les figures, utile.

%\maketitle
%\clearpage

\begin{document}
\bibliographystyle{unsrt}
\nocite{*}

\chapter{Organisation}
\label{cha:Organisation}

\section{Fonctionnement de l'équipe}
\label{sec:Fonctionnement de l'équipe}

Nous avions déjà décidé assez rapidement au début de l'année de Master de former notre groupe pour le projet de programmation. Lors du choix des sujets, nous avons pris en compte les préférences de chacun et nous avons ainsi pu choisir ce sujet. Celui-ci nous a tout de suite intéressé puisque c'était avant tout un projet concret qui vise à répondre à un réel besoin.
En ce qui concerne la mise en place des bases de notre travail en équipe, nous avons réussi à nous organiser rapidement. 
\\
Dès le début, nous avons commencé à répartir nos tâches, ce qui nous a permis d’avancer plus rapidement que ce soit dans l’analyse ou, plus tard, dans la conception. Il était très important pour nous de bien les séparer, car malgré un échange d’informations constant avec des outils tels que Git ou Google Drive, il était parfons difficile de travailler à plusieurs sur une même tâche sans se gêner mutuellement. Les sessions de travail individuelles étaients souvent suivies de réunion, afin de rassembler les travaux de chacun.
\\


\section{Gestion du projet}
\label{sec:Gestion du projet}



\section{Logiciels utilisés}
\label{sec:Logiciels utilisés}

	Git
	Google Drive
	Lucidchart (Google Chrome plugin)
	Textmate

	Technologie ?
	- Html / Css / PHP / JS (Jquery)
	- Python2
	- GTK3


\end{document}
