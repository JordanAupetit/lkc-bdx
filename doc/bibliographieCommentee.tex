\documentclass[16pts]{report}
\usepackage[utf8]{inputenc}
\usepackage[T1]{fontenc}
\usepackage[francais]{babel}
\usepackage{xcolor}
\usepackage[hyphens]{url}
\usepackage{hyperref}
\usepackage{amsmath}
\usepackage{graphicx}
\usepackage{geometry}

\geometry{hmargin=2.5cm,vmargin=1.5cm}

\renewcommand{\thesection}{\arabic{section}}

\begin{document}

\bibliographystyle{unsrt}
\nocite{*}
%\maketitle
%\clearpage

\section{Éléments bibliographiques}
\begin{itemize}
    \item  ~\cite{Waterloo:Etude} est le rapport de l’université de Waterloo
        traitant de l’amélioration des outils de configuration du noyau Linux.
        Celui-ci montre toutes les conclusions de leur étude qui permet de
        commencer ce projet avec une base solide.
    \item  ~\cite{Waterloo:Github} On trouve dans ce dépôt le travail réalisé
        par l’équipe de l’université de Waterloo. Il y a le prototype en java
        qui correspond aux maquettes présentes dans le rapport associé.
    \item L'étude ~\cite{Hubaux:2012:USC:2110147.2110164} montre les principaux
        problèmes rencontrés par des utilisateurs experts, et donc mets en
        valeur nos besoins fonctionnels. En effet, on peut prendre comme exemple
        le manque de suivi dans l'aide à la gestion de conflits entre options.
    \item  ~\cite{Existant:Kernel:tuto} Voici une aide permettant de faciliter
        la démarche pour lancer une compilation. Celle-ci est assez ancienne,
        mais en dehors d’une différence au sujet du Grub, les commandes
        restent exactement les mêmes.
    \item  ~\cite{Existant:Kernel:outils} Voici une seconde aide à la
        compilation qui détaille d’autres points que la précédente.
        Par exemple on peut observer une partie sur la détection du matériel et
        des pilotes du noyau pour générer la configuration.
    \item Les sources de Kconfig ~\cite{Existant:Kconfig:frontends} nous
        permettront de mieux appréhender le mécanisme d'arborescence des options
        présentes dans la configuration d'un noyau Linux. En effet, en reprenant
        ce projet, nous pourrions éventuellement nous abstenir de refaire le
        parsage des nombreux dossiers des sources. De plus les documentations
        ~\cite{Existant:Kconfig:vueDensemble} et ~\cite{Existant:Kconfig:langage}
        contiennent les éléments de syntaxe qui nous ont aidés à comprendre
        comment s'effectue la génération du .config.
\end{itemize}


\bibliography{./bibliographie}

\end{document}
