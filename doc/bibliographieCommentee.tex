\documentclass[16pts]{report}
\usepackage[utf8]{inputenc}
\usepackage[T1]{fontenc}
\usepackage[francais]{babel}
\usepackage{xcolor}
\usepackage[hyphens]{url}
\usepackage{hyperref}
\usepackage{amsmath}
\usepackage{graphicx}
\usepackage{geometry}

\geometry{hmargin=2.5cm,vmargin=1.5cm}

\renewcommand{\thesection}{\arabic{section}}

\begin{document}

\bibliographystyle{unsrt}
\nocite{*}
%\maketitle
%\clearpage

\section{Éléments bibliographiques}
\begin{itemize}
    \item Ce document \cite{Waterloo:Etude} est le rapport de l’université de
        Waterloo traitant de l’amélioration des outils de configuration du
        noyau Linux. On y trouve les résultats de leurs tests auprès de
        différents utilisateurs, ce qui nous permet d’avoir des informations
        sur les besoins réels vis à vis de cet outil.
    \item On trouve dans ce dépôt \cite{Waterloo:Github} le travail réalisé par
        l’équipe de l’université de Waterloo. Il y a le prototype en java qui
        correspond aux maquettes présentes dans le rapport associé.
    \item L'étude \cite{Hubaux:2012:USC:2110147.2110164} montre les principaux
        problèmes rencontrés par des utilisateurs experts, et donc mets en
        valeur nos besoins fonctionnels. En effet, on peut prendre comme exemple
        le manque de suivi dans l'aide à la gestion de conflits entre options.
    \item Le site officiel où trouver le noyau linux \cite{Kernel} est utile
        pour tester la compilation d’un noyau. On peut trouver au sein du
        dossier téléchargé, différents outils permettant de réaliser la
        configuration du noyau avant sa compilation. On y trouve “menuconfig”,
        “xconfig” et “gconfig” qui sont des outils graphiques plus simple
        d’utilisation que l’outil en ligne de commande “config”.
    \item Sur le forum de Linux \cite{Existant:Kernel:ForumTutoConfig} nous
        avons trouvé des explications très détaillées facilitant la
        configuration d’un noyau Linux. Ce guide reprend “pas à pas” chacune
        des options et les détails une à une afin de mieux comprendre ce qui
        peut être activé ou non.
    \item Voici une aide \cite{Existant:Kernel:tuto} permettant de faciliter
        la démarche pour lancer une compilation. Celle-ci est assez ancienne,
        mais en dehors d’une différence au sujet du Grub, les commandes
        restent exactement les mêmes.
    \item Voici une seconde aide \cite{Existant:Kernel:outils} à la compilation 
        qui détaille d’autres points que la précédente.  Par exemple on peut
        observer une partie sur la détection du matériel et des pilotes du noyau
        pour générer la configuration.
    \item L’outil de configuration du système eCos \cite{Existant:EcosConfig}
        réponds à certains problèmes énoncés dans l’étude
        \cite{Hubaux:2012:USC:2110147.2110164}. On remarque la présence d’une
        gestion des conflits de dépendances entre les options choisies absente
        des outils actuels du noyau Linux. En nous en inspirant, on pourra
        éventuellement proposer un mécanisme similaire dans notre outil de
        configuration.
    \item Les sources de Kconfig \cite{Existant:Kconfig:frontends} nous
        permettront de mieux appréhender le mécanisme d'arborescence des options
        présentes dans la configuration d'un noyau Linux. En effet, en reprenant
        ce projet, nous pourrions éventuellement nous abstenir de refaire le
        parsage des nombreux dossiers des sources. De plus les documentations
        \cite{Existant:Kconfig:vueDensemble} et \cite{Existant:Kconfig:langage}
        contiennent les éléments de syntaxe qui nous ont aidés à comprendre
        comment s'effectue la génération du .config. On trouve également sur le
        site officiel du noyau linux des indications
        \cite{Existant:Kconfig:modules} permettant de compiler un module externe
        au sein du noyau. Celui-ci nous permet donc de mieux comprendre le
        fonctionnement des fichiers “Kbuild”.

\end{itemize}


\bibliography{./bibliographie}

\end{document}
