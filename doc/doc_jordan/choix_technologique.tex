\documentclass[16pts]{report}
\usepackage[utf8]{inputenc}
\usepackage[T1]{fontenc}
\usepackage[francais]{babel}
\usepackage{xcolor}
\usepackage[hyphens]{url}
\usepackage[hidelinks]{hyperref}
\usepackage{amsmath}
\usepackage{graphicx}
\usepackage{geometry}
\usepackage{textcomp}
\hypersetup{hypertexnames=true}
\geometry{hmargin=2.5cm,vmargin=1.5cm}

\usepackage{float} %Option H pour les figures, utile.

%\maketitle
%\clearpage

\begin{document}
\bibliographystyle{unsrt}
\nocite{*}

\chapter{Développement}
\label{cha:Développement}

\section{Choix technologique}
\label{sec:Choix technologique}

Dès les premières réunions avec notre client nous avions discuté du langage à 
utiliser pour concevoir l'application. La première proposition était d'utiliser 
le langage C. En effet, vu que cette application a pour but d'être adopté par 
la communauté Linux, il aurait été judicieux de la coder avec le langage 
qu'elle préfère. Dans un premier temps, nous avions pensé à la réaliser 
en Java, mais ce langage n'est pas très apprécié par cette communauté.

Ensuite, après avoir bien avancer dans la recherche de notre existant, nous 
avons trouvé une bibliothèque en Python permettant de manipuler les options 
d'un noyau Linux. Dans une optique d'homogénéité nous avons décidé de réaliser 
toute notre application en Python et cela ne posait pas de problèmes au client.

En ce qui concerne l'utilisation de la bibliothèque Python que nous avons 
trouvé, nous avions le choix entre refaire cette gestion nous même, ce qui 
aurait pris une grande partie de notre temps de développement et celui-ci
était très limité. Nous avons donc décidé de nous servir de cet existant
afin de pouvoir nous concentrer sur d'autres tâches comme l'affichage
des options en conflits.

\end{document}
