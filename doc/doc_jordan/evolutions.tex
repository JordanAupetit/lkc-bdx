\documentclass[16pts]{report}
\usepackage[utf8]{inputenc}
\usepackage[T1]{fontenc}
\usepackage[francais]{babel}
\usepackage{xcolor}
\usepackage[hyphens]{url}
\usepackage[hidelinks]{hyperref}
\usepackage{amsmath}
\usepackage{graphicx}
\usepackage{geometry}
\usepackage{textcomp}
\hypersetup{hypertexnames=true}
\geometry{hmargin=2.5cm,vmargin=1.5cm}

\usepackage{float} %Option H pour les figures, utile.

%\maketitle
%\clearpage

\begin{document}
\bibliographystyle{unsrt}
\nocite{*}

\chapter{Evolutions}
\label{cha:Evolutions}

\section{Evolution des besoins}
\label{sec:Evolution des besoins}

Durant la phase de conception de notre projet, nous nous sommes confronté à 
deux difficultés qui ont fait évoler nos besoins. En effet, nos recherches 
nous pas permis de pouvoir résoudre les conflits automatiquement ni de générer 
une configuration par défaut en fonction du matériel de l'utilisateur, comme
cela a été évoqué précédemment. 
\\
L'objectif de résolution automatique des conflits a été modifié afin de 
simplement aider l'utilisateur à trouver les conflits d'une option. Il se 
chargera de modifier les valeurs des options par lui-même.
\\
En ce qui concerne le besoin de détecter la matériel d'un utilisateur, celui-ci 
s'est transformé pour devenir une base de donnée communautaire. Celle-ci est 
modifiable à l'aide d'un site web. Il est possible d'ajouter des relations 
entre des options et des modules, et entre des modules et du matériels. Les 
modules représentent la partie Driver. Un second besoin était de pouvoir 
ajouter des Tags aux options afin de pouvoir les regrouper afin d'affiner la 
recherche. Ce besoin a également été ajouté sur cette base communautaire.

AJOUTER une photos du SITE

AJOUTER un petit diagramme des classes

\section{Evolution de l'interface}
\label{sec:Evolution de l'interface}

So much screenshots to take..

Expliquer sur chacunes des images les éléments graphiques qui ont changés

\end{document}
