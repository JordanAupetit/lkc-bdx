\documentclass[16pts]{report}
\usepackage[utf8]{inputenc}
\usepackage[T1]{fontenc}
\usepackage[francais]{babel}
\usepackage{xcolor}
\usepackage[hyphens]{url}
\usepackage[hidelinks]{hyperref}
\usepackage{amsmath}
\usepackage{graphicx}
\usepackage{geometry}
\usepackage{textcomp}
\hypersetup{hypertexnames=true}
\geometry{hmargin=2.5cm,vmargin=1.5cm}

%\maketitle
%\clearpage

\begin{document}
\bibliographystyle{unsrt}
\nocite{*}


\chapter{Proposition de tests}
\label{cha:Proposition de tests}

\section{Validité du fichier de configuration}
\label{sec:Validité du fichier de configuration}

Vérifier que le fichier de configuration (.config) soit compatible avec
l’application lors de son chargement. Ce dernier doit respecter le format
d’origine, afin que nous puissions récupérer les informations qui nous sont
utiles.

Vérifier que le fichier de configuration que nous générons via notre outil
puisse être chargé (toujours via notre outil) et qu’il contienne les mêmes
données. On générera à nouveau un fichier de configuration et on pourra
vérifier après ces trois étapes (génération > chargement > génération) si les
informations sont restées les mêmes. Cela montrera que la génération et le
chargement fonctionnent correctement.

\section{Gestion des conflits}
\label{sec:Gestion des conflits}

La sélection d’une option doit signaler à l’utilisateur la présence éventuelle
d’un conflit. Il doit également lui être proposer de pouvoir les résoudre. La
détection d’un conflit se fait en vérifiant que les options actuelles sont
correctes, mais il n’est pas possible de trouver une solution à un conflit en
temps polynomial car c’est un problème NP-Complet (c’est un problème de ce type
car à chaque nouvelle option il faut vérifier les dépendances de toutes les
options précédemment cochées. Il est possible qu’une option précédente exclut
la nouvelle option).

Il faudrait vérifier que les conflits présents dans les fichiers Kconfig du
noyau soient bien affichés dans notre outil.

\section{Système de recherche}
\label{sec:Système de recherche}

Vérifier que la fonction de recherche retourne bien les résultats attendus. Par
exemple, nous pourrons ouvrir un des outils existant (xconfig, gconfig) et
réaliser une recherche. On effectuera la même recherche sur notre outil et on
pourra vérifier si les résultats sont similaires (ce test ne sera valable que
pour les titres des options).

\section{Compatibilité}
\label{sec:Compatibilité}

Si l’utilisateur précise qu’il réalise une configuration pour la machine
courante, l’application devra l’alerter quand celui-ci cherchera à activer une
option incompatible avec son matériel. Par exemple, si sa machine a un
processeur 32-bits et que l’utilisateur sélectionne l’option pour un processeur
64-bits, l’application devra l’avertir que la configuration ne sera pas adaptée
à sa machine.

\section{Interface graphique}
\label{sec:Interface graphique}

Nous allons vérifier la fiabilité de l’interface graphique. Chaque interaction
possible avec les éléments de l’interface doit réaliser l’action attendue.


\end{document}
